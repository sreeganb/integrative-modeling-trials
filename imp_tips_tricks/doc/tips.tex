\documentclass{article}
\usepackage{natbib}
\bibliographystyle{apsrev4-2}

\title{Integrative Modeling Platform Tips and Tricks}
\author{Sree ganesh Balasubramanian}
\date{\today}

\begin{document}
\maketitle

\section{Introduction}
This document provides a collection of tips and tricks for using the Integrative Modeling Platform (IMP).

\section{Parallel execution}
In local machine, for example in pipo, you can use the following command to run the IMP in parallel.
\begin{verbatim}
source ~/work/imp_release/setup_environment.sh
mpirun -np 4 python > output.out 2> error.out &
\end{verbatim}

\section{Working with RMF files}
\subsection{Reading RMF files}
\begin{verbatim}
import os
import IMP
import RMF
import IMP.pmi.output
import IMP.pmi.tools 
from IMP.pmi.tools import get_residue_indexes
from Bio.PDB import PDBParser, PDBExceptions
import matplotlib.pyplot as plt
import numpy as np
import IMP.core

# Load the RMF file
rmf3_file = "./analysis/A_models_cluster2_0.rmf3"
m = IMP.Model()
rh = RMF.open_rmf_file_read_only(rmf3_file)
h = IMP.rmf.create_hierarchies(rh, m)
num_frames = rh.get_number_of_frames()
print("Number of frames in RMF file: ", num_frames)

print("hierarchy", get_residue_indexes(h[0].get_children()[0].get_children()[13]))

mol = IMP.pmi.tools.get_molecules(h[0])
print("molecules", mol)

for i in range(len(mol)):
    if mol[i].get_name() == "Rpt6":
        rpt6_pos = i
    elif mol[i].get_name() == "Rpt2":
        rpt2_pos = i
    else:
        continue
        
sel1 =IMP.atom.Selection(mol[rpt6_pos], residue_index = 222, representation_type = IMP.atom.BALLS, resolution = 1)
parts1 = sel1.get_selected_particles()
coor1 = IMP.core.XYZ(parts1[0]).get_coordinates()
print("coor1", coor1)

sel2 =IMP.atom.Selection(mol[rpt2_pos], residue_index = 258, representation_type = IMP.atom.BALLS, resolution = 1)
parts2 = sel2.get_selected_particles()
coor2 = IMP.core.XYZ(parts2[0]).get_coordinates()
print("coor2", coor2)

print("distance", np.linalg.norm(coor1-coor2))
dist = np.linalg.norm(coor1-coor2)
\end{verbatim}

\section{Specific Techniques}
\subsection{Modeling Protein-Protein Interactions}
\begin{itemize}
    \item ...
\end{itemize}

\subsection{Flexible Modeling}
\begin{itemize}
    \item ...
\end{itemize}

\section{Troubleshooting}
\begin{itemize}
    \item ...
\end{itemize}

\bibliography{references}
\end{document}